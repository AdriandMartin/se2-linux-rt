\documentclass[../main.tex]{subfiles}

\begin{document}

\section{Modificaciones en la distribución}

En este apartado se van a comentar las modificaciones a la distribución de la práctica tres que se tuvieron que realizar para el trabajo.

%----------------------------------------------------------------------------------
\subsection{Kernel}

Dado que se utilizan dos núcleos en el trabajo, y con el fin de evitar tener que renombrarlos cada vez que se quiera usar uno u otro, se han copiado ambos bajo \it{/boot} con los nombres indicados en el apartado 2 de la memoria y se ha creado un enlace simbólico, llamado \it{zImage} (pues es el nombre con el que se espera encontrar el kernel bajo \it{/boot}), a aquel con el cuál se quiere arrancar la distribución.

Esto se hace con el comando siguiente:
\begin{codigo}{bash}
# zImage es solo un enlace simbólico a aquel núcleo con el que se quiere trabajar
ln -s zImage_ftrace_[NFP o FPK] zImage
\end{codigo}

%----------------------------------------------------------------------------------
\subsection{Script de inicialización}

Se editó el script de arranque del sistema (fichero \it{/etc/init.d/rcS}) para que monte el sistema de ficheros de ftrace, \it{tracefs}, en la ruta standard.

Esto supuso añadir, antes de la línea en la que se ejecuta el shell mediante \it{/bin/ash}, lo siguiente:
\begin{codigo}{bash}
# NOTA: esto hace accesible el tracefs en /sys/kernel/debug/tracing
/usr/bin/mount -t debugfs nodev /sys/kernel/debug
\end{codigo}

%----------------------------------------------------------------------------------
\subsection{Instalación de las herramientas}

Una vez compiladas, la instalación de las herramientas consistió en copiar los binarios ejecutables (ficheros \it{tracecmd/trace-cmd} y \it{stress-ng}) en el directorio \it{/usr/bin} de la distribución.

\end{document}
